% Options for packages loaded elsewhere
\PassOptionsToPackage{unicode}{hyperref}
\PassOptionsToPackage{hyphens}{url}
\PassOptionsToPackage{dvipsnames,svgnames,x11names}{xcolor}
%
\documentclass[
  letterpaper,
  DIV=11,
  numbers=noendperiod]{scrartcl}

\usepackage{amsmath,amssymb}
\usepackage{lmodern}
\usepackage{iftex}
\ifPDFTeX
  \usepackage[T1]{fontenc}
  \usepackage[utf8]{inputenc}
  \usepackage{textcomp} % provide euro and other symbols
\else % if luatex or xetex
  \usepackage{unicode-math}
  \defaultfontfeatures{Scale=MatchLowercase}
  \defaultfontfeatures[\rmfamily]{Ligatures=TeX,Scale=1}
\fi
% Use upquote if available, for straight quotes in verbatim environments
\IfFileExists{upquote.sty}{\usepackage{upquote}}{}
\IfFileExists{microtype.sty}{% use microtype if available
  \usepackage[]{microtype}
  \UseMicrotypeSet[protrusion]{basicmath} % disable protrusion for tt fonts
}{}
\makeatletter
\@ifundefined{KOMAClassName}{% if non-KOMA class
  \IfFileExists{parskip.sty}{%
    \usepackage{parskip}
  }{% else
    \setlength{\parindent}{0pt}
    \setlength{\parskip}{6pt plus 2pt minus 1pt}}
}{% if KOMA class
  \KOMAoptions{parskip=half}}
\makeatother
\usepackage{xcolor}
\usepackage[normalem]{ulem}
\setlength{\emergencystretch}{3em} % prevent overfull lines
\setcounter{secnumdepth}{-\maxdimen} % remove section numbering
% Make \paragraph and \subparagraph free-standing
\ifx\paragraph\undefined\else
  \let\oldparagraph\paragraph
  \renewcommand{\paragraph}[1]{\oldparagraph{#1}\mbox{}}
\fi
\ifx\subparagraph\undefined\else
  \let\oldsubparagraph\subparagraph
  \renewcommand{\subparagraph}[1]{\oldsubparagraph{#1}\mbox{}}
\fi

\usepackage{color}
\usepackage{fancyvrb}
\newcommand{\VerbBar}{|}
\newcommand{\VERB}{\Verb[commandchars=\\\{\}]}
\DefineVerbatimEnvironment{Highlighting}{Verbatim}{commandchars=\\\{\}}
% Add ',fontsize=\small' for more characters per line
\usepackage{framed}
\definecolor{shadecolor}{RGB}{241,243,245}
\newenvironment{Shaded}{\begin{snugshade}}{\end{snugshade}}
\newcommand{\AlertTok}[1]{\textcolor[rgb]{0.68,0.00,0.00}{#1}}
\newcommand{\AnnotationTok}[1]{\textcolor[rgb]{0.37,0.37,0.37}{#1}}
\newcommand{\AttributeTok}[1]{\textcolor[rgb]{0.40,0.45,0.13}{#1}}
\newcommand{\BaseNTok}[1]{\textcolor[rgb]{0.68,0.00,0.00}{#1}}
\newcommand{\BuiltInTok}[1]{\textcolor[rgb]{0.00,0.23,0.31}{#1}}
\newcommand{\CharTok}[1]{\textcolor[rgb]{0.13,0.47,0.30}{#1}}
\newcommand{\CommentTok}[1]{\textcolor[rgb]{0.37,0.37,0.37}{#1}}
\newcommand{\CommentVarTok}[1]{\textcolor[rgb]{0.37,0.37,0.37}{\textit{#1}}}
\newcommand{\ConstantTok}[1]{\textcolor[rgb]{0.56,0.35,0.01}{#1}}
\newcommand{\ControlFlowTok}[1]{\textcolor[rgb]{0.00,0.23,0.31}{#1}}
\newcommand{\DataTypeTok}[1]{\textcolor[rgb]{0.68,0.00,0.00}{#1}}
\newcommand{\DecValTok}[1]{\textcolor[rgb]{0.68,0.00,0.00}{#1}}
\newcommand{\DocumentationTok}[1]{\textcolor[rgb]{0.37,0.37,0.37}{\textit{#1}}}
\newcommand{\ErrorTok}[1]{\textcolor[rgb]{0.68,0.00,0.00}{#1}}
\newcommand{\ExtensionTok}[1]{\textcolor[rgb]{0.00,0.23,0.31}{#1}}
\newcommand{\FloatTok}[1]{\textcolor[rgb]{0.68,0.00,0.00}{#1}}
\newcommand{\FunctionTok}[1]{\textcolor[rgb]{0.28,0.35,0.67}{#1}}
\newcommand{\ImportTok}[1]{\textcolor[rgb]{0.00,0.46,0.62}{#1}}
\newcommand{\InformationTok}[1]{\textcolor[rgb]{0.37,0.37,0.37}{#1}}
\newcommand{\KeywordTok}[1]{\textcolor[rgb]{0.00,0.23,0.31}{#1}}
\newcommand{\NormalTok}[1]{\textcolor[rgb]{0.00,0.23,0.31}{#1}}
\newcommand{\OperatorTok}[1]{\textcolor[rgb]{0.37,0.37,0.37}{#1}}
\newcommand{\OtherTok}[1]{\textcolor[rgb]{0.00,0.23,0.31}{#1}}
\newcommand{\PreprocessorTok}[1]{\textcolor[rgb]{0.68,0.00,0.00}{#1}}
\newcommand{\RegionMarkerTok}[1]{\textcolor[rgb]{0.00,0.23,0.31}{#1}}
\newcommand{\SpecialCharTok}[1]{\textcolor[rgb]{0.37,0.37,0.37}{#1}}
\newcommand{\SpecialStringTok}[1]{\textcolor[rgb]{0.13,0.47,0.30}{#1}}
\newcommand{\StringTok}[1]{\textcolor[rgb]{0.13,0.47,0.30}{#1}}
\newcommand{\VariableTok}[1]{\textcolor[rgb]{0.07,0.07,0.07}{#1}}
\newcommand{\VerbatimStringTok}[1]{\textcolor[rgb]{0.13,0.47,0.30}{#1}}
\newcommand{\WarningTok}[1]{\textcolor[rgb]{0.37,0.37,0.37}{\textit{#1}}}

\providecommand{\tightlist}{%
  \setlength{\itemsep}{0pt}\setlength{\parskip}{0pt}}\usepackage{longtable,booktabs,array}
\usepackage{calc} % for calculating minipage widths
% Correct order of tables after \paragraph or \subparagraph
\usepackage{etoolbox}
\makeatletter
\patchcmd\longtable{\par}{\if@noskipsec\mbox{}\fi\par}{}{}
\makeatother
% Allow footnotes in longtable head/foot
\IfFileExists{footnotehyper.sty}{\usepackage{footnotehyper}}{\usepackage{footnote}}
\makesavenoteenv{longtable}
\usepackage{graphicx}
\makeatletter
\def\maxwidth{\ifdim\Gin@nat@width>\linewidth\linewidth\else\Gin@nat@width\fi}
\def\maxheight{\ifdim\Gin@nat@height>\textheight\textheight\else\Gin@nat@height\fi}
\makeatother
% Scale images if necessary, so that they will not overflow the page
% margins by default, and it is still possible to overwrite the defaults
% using explicit options in \includegraphics[width, height, ...]{}
\setkeys{Gin}{width=\maxwidth,height=\maxheight,keepaspectratio}
% Set default figure placement to htbp
\makeatletter
\def\fps@figure{htbp}
\makeatother

\KOMAoption{captions}{tableheading}
\makeatletter
\@ifpackageloaded{tcolorbox}{}{\usepackage[many]{tcolorbox}}
\@ifpackageloaded{fontawesome5}{}{\usepackage{fontawesome5}}
\definecolor{quarto-callout-color}{HTML}{909090}
\definecolor{quarto-callout-note-color}{HTML}{0758E5}
\definecolor{quarto-callout-important-color}{HTML}{CC1914}
\definecolor{quarto-callout-warning-color}{HTML}{EB9113}
\definecolor{quarto-callout-tip-color}{HTML}{00A047}
\definecolor{quarto-callout-caution-color}{HTML}{FC5300}
\definecolor{quarto-callout-color-frame}{HTML}{acacac}
\definecolor{quarto-callout-note-color-frame}{HTML}{4582ec}
\definecolor{quarto-callout-important-color-frame}{HTML}{d9534f}
\definecolor{quarto-callout-warning-color-frame}{HTML}{f0ad4e}
\definecolor{quarto-callout-tip-color-frame}{HTML}{02b875}
\definecolor{quarto-callout-caution-color-frame}{HTML}{fd7e14}
\makeatother
\makeatletter
\makeatother
\makeatletter
\makeatother
\makeatletter
\@ifpackageloaded{caption}{}{\usepackage{caption}}
\AtBeginDocument{%
\ifdefined\contentsname
  \renewcommand*\contentsname{Table of contents}
\else
  \newcommand\contentsname{Table of contents}
\fi
\ifdefined\listfigurename
  \renewcommand*\listfigurename{List of Figures}
\else
  \newcommand\listfigurename{List of Figures}
\fi
\ifdefined\listtablename
  \renewcommand*\listtablename{List of Tables}
\else
  \newcommand\listtablename{List of Tables}
\fi
\ifdefined\figurename
  \renewcommand*\figurename{Figure}
\else
  \newcommand\figurename{Figure}
\fi
\ifdefined\tablename
  \renewcommand*\tablename{Table}
\else
  \newcommand\tablename{Table}
\fi
}
\@ifpackageloaded{float}{}{\usepackage{float}}
\floatstyle{ruled}
\@ifundefined{c@chapter}{\newfloat{codelisting}{h}{lop}}{\newfloat{codelisting}{h}{lop}[chapter]}
\floatname{codelisting}{Listing}
\newcommand*\listoflistings{\listof{codelisting}{List of Listings}}
\makeatother
\makeatletter
\@ifpackageloaded{caption}{}{\usepackage{caption}}
\@ifpackageloaded{subcaption}{}{\usepackage{subcaption}}
\makeatother
\makeatletter
\@ifpackageloaded{tcolorbox}{}{\usepackage[many]{tcolorbox}}
\makeatother
\makeatletter
\@ifundefined{shadecolor}{\definecolor{shadecolor}{rgb}{.97, .97, .97}}
\makeatother
\makeatletter
\makeatother
\ifLuaTeX
  \usepackage{selnolig}  % disable illegal ligatures
\fi
\IfFileExists{bookmark.sty}{\usepackage{bookmark}}{\usepackage{hyperref}}
\IfFileExists{xurl.sty}{\usepackage{xurl}}{} % add URL line breaks if available
\urlstyle{same} % disable monospaced font for URLs
\hypersetup{
  pdftitle={Nickel Bingo},
  pdfauthor={Mark Niemann-Ross},
  colorlinks=true,
  linkcolor={blue},
  filecolor={Maroon},
  citecolor={Blue},
  urlcolor={Blue},
  pdfcreator={LaTeX via pandoc}}

\title{Nickel Bingo}
\author{Mark Niemann-Ross}
\date{}

\begin{document}
\maketitle
\ifdefined\Shaded\renewenvironment{Shaded}{\begin{tcolorbox}[interior hidden, frame hidden, breakable, borderline west={3pt}{0pt}{shadecolor}, sharp corners, boxrule=0pt, enhanced]}{\end{tcolorbox}}\fi

\hypertarget{a-new-home}{%
\subsection{A New Home}\label{a-new-home}}

At 63 years old, I'm the youngest community member of my Mother-In-Law's
senior residence hall. How I got here is an interesting story, but one
for later. It's enough to know I'm helping her recover her strength.
Unexpectedly, I have access to a wealth of new perspectives from an
older generation.

This is the generation who told me I needed a haircut, who told me Satan
inspired the Beatles, and who told me astronauts drank Tang. It's also
the generation that changed my diapers, let me use the family car, and
occupy most seats of our senior levels of government.

Every Sunday, a weekly activity guide appears in my mailbox. Fitness
Class is Monday at 1:00 p.m. The Catholics among us recite The Rosary at
2:30 on Wednesdays. Ice Cream is just before the Rosary. Pub night is in
the evening at 6:00.

That's all very nice, but how we \emph{long} for bingo night.

\hypertarget{two-important-things}{%
\subsection{Two Important Things}\label{two-important-things}}

\emph{Nickel Bingo is every Thursday night at 6:00 p.m.}

There are two notable things about this sentence. Least of the two, but
still significant, is the time: Nickel Bingo happens at 6 p.m.
Activities occurring past 5:00 p.m. are \emph{late-evening}.
\sout{Dinner} Supper is served from 4:30 to 5:30 p.m.; past then, the
halls become the turf of the night owls. Any resident gatherings after
six p.m. are surreptitious. Walk unannounced into one of the activity
rooms and you may find residents quietly playing Rummikub, but that's
just to avert suspicion; when you move out of earshot, they will return
to plotting world domination.

Previous to living here, these late-night residents would have huddled
around the corner table in a seedy bar, speaking in lowered voices and
pausing their negotiations when the server comes by to refill drinks.
After living here for a month, I've gained a bit of trust with them. But
they still change conversation when I approach.

At six o'clock, the sharks circle the silent waters of my residence
hall. At six o'clock on Thursday evenings, the sharks forego Rummikub
and swim to the Independent Dining Room for \emph{nickel bingo}.

\hypertarget{nickel-bingo}{%
\subsection{Nickel Bingo}\label{nickel-bingo}}

Nickel Bingo is the second of notable things in the above sentence.
There are \emph{other} bingo games during the week, but candy bingo and
jewelry bingo are minor-league compared to nickel bingo. The bingo
sharks take part, but only as a side to the big event on Thursday
evening. If you win at candy or jewelry bingo, you get to choose a piece
of candy (or a piece of jewelry). But, \emph{you don't have to put
anything in the cup to play}. For these games, you can show up
empty-handed and still win. Not so with nickel bingo.

Very few men take part in candy or jewelry bingo. I have asked the
activities coordinators if there is ever gun bingo, fishing lure bingo,
or whiskey bingo. (answer: \emph{no!}) I've thought of organizing an
underground bingo association for the men; the poorly lit basement
parking garage would be ideal for a cigar-smoking, whiskey-drinking,
Friday night game. So far, I have abstained from instigating. I don't
want to be the cause (or subject) of an \emph{official memo} on
appropriate evening activities.

\begin{tcolorbox}[enhanced jigsaw, bottomrule=.15mm, title={Official Memos}, colbacktitle=quarto-callout-note-color!10!white, left=2mm, leftrule=.75mm, coltitle=black, arc=.35mm, breakable, rightrule=.15mm, opacityback=0, bottomtitle=1mm, colframe=quarto-callout-note-color-frame, toprule=.15mm, toptitle=1mm, titlerule=0mm, opacitybacktitle=0.6, colback=white]
Official memos come from the Facility Director. They are copied on
letterhead and appear in your campus mail slot. The instigating resident
or precipitating situation is only vaguely insinuated, but the intent is
clear: \emph{``Keep it up, buster, and your children will be
interviewing other residence halls for your next placement.''}
\end{tcolorbox}

\hypertarget{bingo-basics}{%
\subsection{Bingo Basics}\label{bingo-basics}}

Knowing the essentials of bingo will help you understand the drama
associated with nickel bingo. I'll make this as brief as possible, after
which I'll explain why our bingo night frequently descends into mayhem.

Simply explained, each bingo player possesses one or more printed cards
displaying a grid of random numbers. The bingo caller sits in front of
the room and announces a series of random numbers. When a called number
matches a number on the player card, the player marks this position on
the grid. When the marked positions on a bingo card align in a pattern
(such as a complete row or column) the player calls out ``BINGO.''
Assuming all the player numbers match caller numbers, the player is a
winner. Everyone calls out in amazement and the game is reset.

The random numbers are traditionally generated with a bucket of numbered
balls, although some callers use computer programs. With the
bucket-of-balls approach, the caller shuffles the balls (either by hand
in a bucket or with a tumbler cage), retrieves a ball, then reads the
value printed on the ball. Computer applications select a random number
at the push of a button, then display it on screen. An internet search
will turn up several versions for all phones and computers.

Until recently, I did not understand there are different versions of
bingo. The number of balls used to generate random numbers can vary from
thirty to ninety. The rows and columns on the cards, when multiplied,
can equal as few as nine positions and as many as twenty-five. At this
residence hall, we play American bingo; seventy-five balls for random
number generation and the players use cards with a five-by-five matrix
of integers. The letters of ``BINGO'' top each column. The center square
is marked as ``FREE.''

If you want to play, you select a bingo card from a stack. I don't
believe you can bring your own card, although I haven't tried. This
isn't like pool where you bring your custom-made stick. At the least, it
would be a topic of conversation for weeks; at worst, it would be the
topic of another \emph{official memo}.

Our bingo cards are reusable; each number appears in a window with a
sliding, transparent-red piece of plastic. When a number is called, you
slide the plastic down. When the game is over, you slide the plastic to
the top of each window and prepare for the next round. With age, these
plastic sliders become sticky and reticent to move. I have found no
maintenance instructions for these cards. I've tried to disassemble a
card for better access to the internals, but like carburetors and mobile
phones, I don't have the proprietary tool and my many attempts have only
resulted in fewer operational bingo cards.

When I am less engaged in the game, my mind wanders through the
wonderful math of these cards. There are algorithmic rules for placing
numbers on the board. Each column has a range of discrete integers, each
integer can only appear once. The integers must be random in value and
occurrence.

I shared my rapture of the math behind these cards with Janell and she
gave me a look I have received only once before. In that case, I had
excitedly discussed my progress in creating an etch-a-sketch program for
the Apple //e computer, written in assembler. We were dirt-poor and she
failed to see how this could be a career-enhancing pursuit. (Side note:
It \emph{was} career enhancing, but that's another story.)

As much as I would like to inflict a comprehensive discussion of the
programming necessary to generate the bingo matrix, I realize the
audience for this sort of thing is limited. Instead, I have added an
appendix with code to create a proper bingo card. You're welcome.

\hypertarget{the-player-table}{%
\subsection{The Player Table}\label{the-player-table}}

Our bingo sharks have unofficially agreed on assigned seating, although
this was discouraged by was an \emph{official memo}. I advise visitors
to sit at a table in the room's periphery to avoid crashing the normal
order; even then, you may have to crowd in as the fifth at a table of
four. Maybe you should consider just watching, rather than causing a
disruption.

Because of my age, I am an honored guest and am invited to join one of
the regular tables. Each of us has two bingo cards. (I could play more,
but two seems to be the norm. And I've learned to stick to the norm.
Mostly. Occasionally. \emph{It's complicated}.)

If I forget to bring my jar of nickels, I will sponge off my
Mother-in-law. She's gracious, but I worry this will strain our
relationship. Her jar contains quarters, dimes, pennies, and sometimes
dollar bills. I'll explain in a minute, but non-nickels cause confusion.
For God's sake, don't bring foreign currency: This would be probable
cause for yet another \emph{official memo}.

A small bowl is in the middle of our table. It might be empty or it
might contain change. I want to play this round of bingo so I contribute
one nickel per bingo card. Two bingo cards, two nickels. This is why we
call this game nickel bingo. Four people at the table times two bingo
cards will cause eight nickels in the bowl. Which equals forty cents.
Instead of two nickels, we are allowed to deposit one dime. If someone
deposits a quarter, they can remove fifteen cents from the bowl.

Before I remove money from the bowl, I am \emph{inordinately} clear
about my action and intention. Someone might misconstrue sudden movement
near the group nickel bowl. Fortunately, weapons are not allowed in our
building. But I don't want to be shunned.

\hypertarget{game-phase-one-the-collection}{%
\subsection{Game Phase One-The
Collection}\label{game-phase-one-the-collection}}

Before each round of bingo can begin, the designated money collector
circulates among the tables and gathers the change from each bowl. We
assume the money collector confirms each table has contributed the
correct amount (\emph{but maybe not. More on this in a minute}). This
collection is placed at the head table and the bingo caller can proceed.

Collecting bingo fees must be done with efficiency. Our sharks are
waiting for the game to begin, but not patiently. Pick up the bowl,
count the change. This should be simple enough.

\emph{But not so fast\ldots{}}

You might think the job of money collection is inconsequential, but it
is not. An informal and self-designated sub-committee of bingo sharks
bestows the honor of money collector upon an individual. The
qualifications are not available for public review anywhere that I've
seen, but I believe the main qualifier is squeezing between tightly
packed tables, walkers, and wheelchairs. This requires the ability to
bend your knees and locomote without support. I am proud to note I have
been asked to perform this task and have discharged my duties with the
earnestness of a first-time student council president.

Now-I propose a conundrum for you to consider. \emph{What is the proper
response if the change is not correct for the number of bingo cards
visible on the table?}

My first strategy might be to ask if everyone has contributed their two
nickels. When I'm fortunate, one member of the table will realize it is
time to ante-up and contribute their fair share. Unfortunately, this
rarely happens.

There are two complexities to consider, possibly more. For example, this
table may assume that nickels are placed in the cup \emph{after} the
money collector has emptied the bowl for the current round. This is an
advanced bingo shark technique. Speaking from personal experience, it is
easy to become confused about the state of the bowl and you would be
mistaken to blame this on any of the seven stages of dementia.

It's also possible there is a change-making anomaly, one indicator being
the presence of a quarter in the bowl and exacerbated by the cumulative
arthritis of everyone at the table. Arthritis is when your knuckles
grind bone-on-bone. To develop empathy, insert 60-grit sandpaper between
each joint in your fingers so it is excruciating to bend them, then wrap
each finger in duct-tape so you cannot flex. Now pick up three nickels
placed in a small bowl to make change for your quarter.

During my training to become certified as a bingo money collector, I
also learned of the inadvertent bowl collection confusion. Here's a
scenario:

\begin{enumerate}
\def\labelenumi{\arabic{enumi}.}
\item
  I arrive at the table and develop empathy with the sharks at this
  table. I assure them they will win during this round. They assume I am
  lying to gain their favor, which is true. I'm trying to make friends
  here.
\item
  Someone will need change, most likely for a dollar. I count out
  nickels from the bingo fees I've already collected, then exchange
  those nickels for the dollar.
\item
  I reach to collect the bowl from the table. Sometimes, someone will
  object; ``You've already collected for this round.''
\end{enumerate}

I may think I am \emph{certain} I have not collected. But I am only one
vote out of five, the other four being the players seated at the table.
They may assume the bowl contains money for the next round. The act of
making change may have interrupted the normal flow of the interaction
between sharks and collector. Or sadly, there may be some dementia and
short-term memory is a bit wobbly.

Untangling an error in cash flow is tortuous. I've tried to reason
through a narrative of the collection, but the accounting trail is
miserable. I've found it easiest to just agree everything is accurate,
then pour this bowl into the general collection. This isn't an IRS audit
and the amount to be won is secondary to the win itself. Smile and move
on.

At times, someone may be unable to contribute or is flustered with the
quick flow of money across the table. I've learned to carry a
supplemental pocket of nickels and make a lightning loan where
necessary. \emph{``I've got you covered,''} I tell my new friend. In a
few minutes everyone forgets about the loan, but the friendship remains.
I count that as a win.

Collections has become an issue of major contention and even made it to
a topic of discussion at the recent town hall. Two volunteers stepped up
with a plan; now we have a collection system inspired by the traditions
of the Lutheran Church. When God ordains, things go smoother.

\hypertarget{game-phase-two-the-calling}{%
\subsection{Game Phase Two-The
Calling}\label{game-phase-two-the-calling}}

After money is collected, the bingo caller proceeds with the
announcement of the numbers. Again, let me explain the simple concept,
then I'll explain the nuances that lead to shouting.

\begin{tcolorbox}[enhanced jigsaw, bottomrule=.15mm, title={Bingo Caller Certification}, colbacktitle=quarto-callout-note-color!10!white, left=2mm, leftrule=.75mm, coltitle=black, arc=.35mm, breakable, rightrule=.15mm, opacityback=0, bottomtitle=1mm, colframe=quarto-callout-note-color-frame, toprule=.15mm, toptitle=1mm, titlerule=0mm, opacitybacktitle=0.6, colback=white]
You might think I am kidding about bingo caller certification. I am not.
Depending on your state of residence, you may be required to take a
class, pass a test, and/or annually renew your license. Search the
internet for \emph{bingo certification} for details.
\end{tcolorbox}

The caller withdraws a ball from a bucket of 75 balls and announces the
letter/number combination (``G-48. That's GEE\ldots FOUR\ldots EIGHT),
then-and this is important-place the ball in the correct place in a
master matrix with seventy-five depressions. Each hole is numbered and
arranged in five columns of fifteen rows. Each column is labeled with
one letter from the word''BINGO.'' With the ball safely in place, the
caller withdraws another ball and repeats the process. The master matrix
is an easy way to confirm which numbers have been called. Which is
really important when someone calls\ldots{}

\textbf{BINGO!}

\hypertarget{what-could-possibly-go-wrong}{%
\subsubsection{What could possibly go
wrong?}\label{what-could-possibly-go-wrong}}

Before we get to the climax of the evening, I'll discuss the reality of
a live bingo game in the context of a simplified communication model.
There are many versions advanced and researched by communications
specialists; I'll just consider the sender, noise, and the receiver.

\hypertarget{sender}{%
\paragraph{Sender}\label{sender}}

Qualities of a good bingo caller - the sender, in this discussion -
include enunciation, volume, and pacing - much like an auctioneer, but
slower. There is such a thing as lightning bingo-but not here.

Ideally, the bingo caller will draw a ball from the bucket, then pause
to look at the letter/number on the ball, call out the number clearly
and carefully, pause, repeat the letter/number, place the ball in the
master matrix, pause, shuffle the balls, and draw the next ball.

However, some callers have forgotten their training and acquire bad
habits: Speaking softly. Sloppy pronunciation. Calling too fast. When
auditioning a caller, we consider experience as pastor, square-dance
caller, sports announcer, elementary school teacher, or game-show host.
Unfortunately, the qualification of the caller is not the only chaotic
force in action.

Pacing is every bit as important as volume and enunciation. My brain is
less than seventy years old and I don't have enough empathy for the
reduction in processing speed as we age. When I was twenty-years-old, I
could do polynomial equations in my head. I could also perform deep-knee
bends while lifting seventy-pound weights. Forty years later, math takes
longer and my knees support less weight. Pausing between numbers gives
everyone a chance to hear the number, parse the number, check both bingo
cards, slide the plastic cover on a number, wiggle the cover if it
sticks, confirm the number with a neighbor, check for a bingo pattern,
and re-focus in preparation for the next number.

\begin{tcolorbox}[enhanced jigsaw, bottomrule=.15mm, title={Polynomial Equations}, colbacktitle=quarto-callout-note-color!10!white, left=2mm, leftrule=.75mm, coltitle=black, arc=.35mm, breakable, rightrule=.15mm, opacityback=0, bottomtitle=1mm, colframe=quarto-callout-note-color-frame, toprule=.15mm, toptitle=1mm, titlerule=0mm, opacitybacktitle=0.6, colback=white]
This is an example of a simple polynomial equation: \(x^2 + 2x +5\)

This is a complex polynomial equation: \(a_2x^2 + a_1x + a_0\)

Quickly solve for \(x\) . Good luck\ldots{}
\end{tcolorbox}

I imagine a professor calling out polynomial equations at the pace of
one per minute. I might solve the first one. Possibly the second. By the
third I would call for a pause, and by the fourth I would be
irretrievably lost and complaining to my table-mates. Madness and
shouting would ensue. We don't like it when bingo callers move too fast.

\hypertarget{noise}{%
\paragraph{Noise}\label{noise}}

Communication models include the channel of transmission (paper for
writing, copper wires for phone calls, air for sound) and include noise
as a complicating factor. Smeared ink for paper, static for phone calls,
and rattling bingo balls for bingo callers.

About those damn bingo balls. They are plastic; every time the caller
stirs the bucket of balls, they generate noise akin to a ball-bearing
factory. It masks human speech. Me and the bingo sharks tell callers not
to stir the balls while calling a number. If they don't pause long
enough between calling the number and stirring the balls, we hear the
following\ldots{}

\begin{quote}
``O-63\ldots that's O- (RATTLERATTLERATTLE). The next number is B-3.
That's B- (RATTLERATTLERATTLE).''
\end{quote}

RATTLERATTLERATTLE results in secondary fallout noise, caused by one or
more calls to repeat the last number, followed by 25\% of the bingo
players repeating the number (out of sync, so the response is also
confused) followed by the bingo caller demanding silence, which is
followed by another 25\% of bingo players again asserting the latest
number simultaneous to a different 25\% asking if it was a different
number, followed by the caller repeating the number, followed by someone
asking if that is the last number or the next number. There is
background rumbling about the caller being inaudible or the players
needing to pay better attention. I suspect this is the reason we cannot
keep bingo callers for more than a couple of months.

Also problematic is outsiders: relatives, children, nursing staff,
janitors. Relatives and children are the worst, as they have no respect
for the sanctity of the game in progress. Typically, they appear at the
door of the bingo parlor, squeal, and call out; ``GRANDMA! WE'RE HERE!''
They invade our bingo sanctuary en masse, surround their relative and
talk loudly about their \emph{car trip} and \emph{how good it is to see
them} and \emph{how big the kids have grown} and \emph{we have this pie
we baked for you} and \emph{what's going on here} and \emph{Oh, I love
bingo} and \emph{we should join in}.

I'm enthusiastic about visitors. They brighten up everyone's day. But
\emph{gawdammit}, we're trying to play \textbf{bingo} here.
\textbf{\emph{Nickel Bingo!}}

Likewise, the nursing staff will sometimes stop by for blood pressure
and pulse ox. They truely are heroes and wrangle medications and blood
pressures with grace and care. But the brief interruption in game play
means possibly missed numbers, causing one or more calls to repeat the
last number, followed by 25\% of the bingo players repeating the number
\ldots{} and so on.

If maintenance is working late, they may vacuum in the next room. A
vacuum cleaner produces something called ``white noise,'' and it masks
the bingo caller's voice. Players won't be able to hear, one or more
calls to repeat the last number \ldots{} and further so on and so forth.

\hypertarget{receiver}{%
\paragraph{Receiver}\label{receiver}}

If the caller enunciates, projects, and paces-and someone's kid isn't
playing with a vacuum cleaner or banging out chopsticks on the piano,
then it is reasonable to assume everything will go smoothly. However,
there is still the matter of the receiver. Me. The bingo player, in this
scenario.

By age 80, it's not uncommon to have a hearing loss of 50\% in the
frequency range of the human voice. Roughly one-third of the bingo
players have some level of dementia, one symptom being confusion and
agitation when dealing with the cacophony of many people talking at
once. My friends are doing their best, but they need a bit of slack.

Do you wonder how this feels? Did you solve the polynomials shown above?
Solve them again while teaching a first-year band class. This is an
example of cacophony as applied to bingo.

By the way, good table mates look out for each other. After I scan my
bingo cards, I stealthily scan my neighbor's cards and politely point
out any numbers they may have missed. Done quietly, this is a gesture of
kindness.

\hypertarget{game-phase-three-bingo}{%
\subsection{Game Phase Three-BINGO}\label{game-phase-three-bingo}}

\textbf{BINGO!}

It eventually happens. Sometimes I hear it from another table. Sometimes
from my table. Sometimes I double-take at my card, realize I have the
game pattern and hear myself yell \textbf{BINGO}.

In American bingo, there are three basic winning patterns: Five in a
row, five in a column, and five diagonally. We have three additional
patterns: four corners, picture frame, and the high-stakes
\emph{blackout}. (more about blackout in a minute) Some callers try
additional patterns (i.e.~postage stamp, two lines, small diamond) but
approval among the sharks is mixed.

Upon hearing bingo, the caller pauses the game and confirms the winner.

An experienced caller will remind players \emph{not} to clear their
board in case bingo isn't confirmed. An inexperienced caller will forget
this step and players will clear their boards. If a bingo is not
confirmed, the confusion quotient of the room will rise. A player whom
mistakenly cleared their board will request a re-call of all the drawn
numbers, but that takes time and is not a popular suggestion. By the
way, not clearing your bingo card until confirmation is an earmark of an
experienced player. If you wait until confirmation, congratulations, you
are on your way to bingo sharkdom.

Confirming a bingo is mechanically easy, but can be problematic. If
everything runs smoothly, the winning player reads out-loud the numbers
comprising their bingo, the caller confirms those numbers were actually
called and if all goes well, the caller confirms a winning bingo. To the
winner goes the nickels and everyone resets their bingo cards.

Sometimes it isn't smooth. As in all steps of bingo, problems appear
during the implementation. The concept of ``out loud'' has various
interpretations with multiple variables, including speed of delivery and
volume of the player. Here's a scenario: When I claim bingo, I call out
the five winning numbers. I can do this as quickly and as softly as I
want. (for example, \emph{8-16-Free-57-72}. ``Free'' because this is the
middle row) I might include the corresponding B-I-N-G-O letters.

Callers check those numbers against the matrix of balls used to generate
the calls. Maybe the caller's hearing isn't as good as it used to be.
Maybe other players are chatting/gossiping/remarking at how close
\emph{they} were at winning. Maybe the caller's memory for numbers isn't
100\%.

Confirming a bingo often takes multiple tries. But in the end (usually)
everything works out and we have a winner.

\hypertarget{blackout-bingo}{%
\subsubsection{Blackout Bingo}\label{blackout-bingo}}

Each evening bingo session includes ten rounds with the standard
row/column/diagonal bingo pattern. As mentioned above, these games cost
one nickel per card. There is an additional, final round which uses a
blackout pattern - cover \textbf{all} windows on the card. It surprised
me to learn the cost per card doubles to \emph{ten cents per card}.
Where I put down five cents per card times two cards per player times
four players per table (equals forty cents) I now put down \emph{ten}
cents per card (equals \emph{eighty} cents per table). Our bingo room
has space for seven tables, so winning blackout bingo is a big payout -
over FIVE DOLLARS!

Blackout bingo takes longer than standard patterns, but the suspense is
\emph{killer!} I feel my blood pressure rise as I watch the card
approach 100\% coverage. Several times I've had only two, maybe one
window open. Several times my fantasy of a big payout has been crushed
by someone else calling \textbf{BINGO}. \emph{Dammit!}

\hypertarget{game-phase-four-payout}{%
\subsection{Game Phase Four-PAYOUT}\label{game-phase-four-payout}}

I've won several times! It tempted me to do a victory dance, beat my
chest, throw my winning bingo card across the room and shout ``\textbf{I
DOMINATE! I AM THE BINGO JEDI MASTER! COWER IN FEAR, ALL WHO BEHOLD
ME!}''

I haven't done this. My knees are not up to dancing around. My older
table mates would be ill-advised to beat their chests, considering any
history of heart trouble, and throwing anything across the room is
likely to poke someone's eye out.

When I win, I revel in the momentary endorphin rush of seeing a pile of
change pour into my bingo change jar, I tell my table friends how much
fun that was and how surprised I am to have won. Then I get ready for
the next round.

\hypertarget{two-or-more-winners}{%
\subsubsection{Two or More Winners}\label{two-or-more-winners}}

Normally, one single person wins each round, but it's not uncommon for
two (or more) people to win simultaneously. (The odds of winning bingo
is a fascinating subject if you are a statistician or math professor.
Look it up on the internet. The rest of us should leave the math geeks
in peace).

When multiple \textbf{BINGO!}s are confirmed, winners split the prize.
As a money collector, it is my job to split the cash. There are three
strategies I've used to accomplish this accounting feat of daring-do.

\begin{itemize}
\item
  Total the money, then divide by the number of winners. Count out a
  pile for each winner.
\item
  Drag into piles. One for you. One for me. Repeat until all coins are
  distributed.
\item
  Eyeball. Hmm\ldots this pile looks about this big. Some for one winner
  and some for the other winner.
\end{itemize}

There's a secret I learned and I'll tell you if you promise not to make
a big damn deal about it.

Here's the secret: Where I live, \emph{Nobody cares!}

I divide up the money in mostly equal piles, then give it to the
winners. I pour that money into a pouch or jar already containing an
uncounted amount of change; there is no accounting trail or software
involved. The thrill of the win counts. Delaying the start of the next
game is a buzz kill.

\hypertarget{final-thoughts}{%
\subsection{Final Thoughts}\label{final-thoughts}}

My family played cards. The intent was to give your hands something to
do while you talked. Games were a second act after family meals; we
enjoyed hanging out together (most of the time) and we needed some sort
of excuse to do so.

I play bingo with my friends. Some of them are losing their memory and
reasoning and physical dexterity. All of them are getting older. The
best part of playing bingo is hanging out with each other during the
short time we have.

\hypertarget{appendix}{%
\subsection{Appendix}\label{appendix}}

Here is R code to build an American bingo card. R is a programming
language used by statisticians and researchers. It is overkill for
generating bingo cards, but it amuses me.

\begin{Shaded}
\begin{Highlighting}[]
\NormalTok{gimmeFive }\OtherTok{\textless{}{-}} \ControlFlowTok{function}\NormalTok{(range) \{}
  \FunctionTok{as.character}\NormalTok{( }\FunctionTok{sample}\NormalTok{(}\AttributeTok{x =}\NormalTok{ range, }\AttributeTok{size =} \DecValTok{5}\NormalTok{))}
\NormalTok{\}}

\NormalTok{BingoMatrix }\OtherTok{\textless{}{-}} \FunctionTok{matrix}\NormalTok{(}\AttributeTok{data =} \FunctionTok{c}\NormalTok{(}\FunctionTok{gimmeFive}\NormalTok{(}\DecValTok{1}\SpecialCharTok{:}\DecValTok{15}\NormalTok{),}
                               \FunctionTok{gimmeFive}\NormalTok{(}\DecValTok{16}\SpecialCharTok{:}\DecValTok{30}\NormalTok{),}
                               \FunctionTok{gimmeFive}\NormalTok{(}\DecValTok{31}\SpecialCharTok{:}\DecValTok{45}\NormalTok{),}
                               \FunctionTok{gimmeFive}\NormalTok{(}\DecValTok{46}\SpecialCharTok{:}\DecValTok{60}\NormalTok{),}
                               \FunctionTok{gimmeFive}\NormalTok{(}\DecValTok{61}\SpecialCharTok{:}\DecValTok{75}\NormalTok{)),}
                      \AttributeTok{nrow =} \DecValTok{5}\NormalTok{)}

\FunctionTok{colnames}\NormalTok{(BingoMatrix) }\OtherTok{\textless{}{-}} \FunctionTok{c}\NormalTok{(}\StringTok{"B"}\NormalTok{,}\StringTok{"I"}\NormalTok{,}\StringTok{"N"}\NormalTok{,}\StringTok{"G"}\NormalTok{,}\StringTok{"O"}\NormalTok{)}

\NormalTok{BingoMatrix[}\DecValTok{3}\NormalTok{,}\DecValTok{3}\NormalTok{] }\OtherTok{\textless{}{-}} \StringTok{"Free"}

\NormalTok{BingoMatrix}
\end{Highlighting}
\end{Shaded}

\begin{verbatim}
     B    I    N      G    O   
[1,] "14" "29" "37"   "49" "67"
[2,] "2"  "22" "34"   "60" "75"
[3,] "12" "19" "Free" "57" "61"
[4,] "6"  "21" "31"   "55" "74"
[5,] "3"  "24" "32"   "59" "69"
\end{verbatim}



\end{document}
